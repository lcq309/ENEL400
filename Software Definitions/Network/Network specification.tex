\documentclass[a4paper,12pt]{article}

\usepackage{geometry}
\geometry{margin=1in}

\usepackage{enumitem}
\usepackage{graphicx}
\usepackage{float}

\begin{document}
	
	\title{RS485 Wired Network Specification}
	\author{Your Name}
	\date{\today}
	
	\maketitle
	
	\section{Introduction}
	
	This document provides a detailed specification for the RS485-based wired network. It outlines the components, protocols, and configurations necessary for the successful implementation and operation of the network.
	
	\section{Network Topology}
	
	Physical layout and topology:
	
	\subsection{Wireless topology}
	The wireless network will be an xbee mesh using zigbee modules.
	
	\subsection{Wired network}
	The wired network is based on the RS485 physical standard.
	The wired network will use a bus topology, with all devices connected in series and terminated using 120$\Omega$ resistors.
	
	\section{Hardware Components}
	
	List and describe the hardware components involved in the network:
	\subsection{wireless}
	The wireless network is based on our xbee modules.
	\subsection{wired}
	The wired network is based on the RS485 standard, with our own handling for messages.
	\subsubsection{Network devices}
	The network consists of one Master device, with up to 255 subordinate devices in a single bus.
	\subsubsection{RS485 transceiver}
	The selected transceiver allows for up to 16mbps communication, and up to 255 receiving devices on a network.
	(include a reference link to the RS485 transceiver)
	\subsubsection{power supply}
	The master device is capable of supplying up to 11w on each port.
	
	\subsubsection{Cables and connectors}
	The expected connector is a generic RJ45.
	With cables ranging from cat5e to cat6.
	A single twisted pair is for data transmission, the other three are for power.
	
	\subsubsection{Termination resistors}
	The bus should be terminated with a 120$\Omega$ resistance on each end.
	
	\section{Communication Protocol}
	
	\subsection{Wireless}
	the wireless network is dependent on the built in protocols of the xbee modules.
	
	\subsection{Wired}
	Detail the communication protocol used on the RS485 network. Include information on:
	
	\subsubsection{Baud rate}
	
	we are expecting a bit rate of around 115200.
	
	\section{Data Frames}
	
	Describe the format of data frames used in the RS485 network. Include details such as:
	
	\begin{itemize}[label=--]
		\item Start and stop bits.
		\item Data length.
		\item Addressing scheme.
		\item Payload structure.
	\end{itemize}
	
	\section{Electrical Characteristics}
	
	Specify the electrical characteristics of the RS485 network:
	
	\begin{itemize}[label=--]
		\item Voltage levels.
		\item Maximum cable length.
		\item Maximum number of nodes.
		\item Termination resistor values.
	\end{itemize}
	
	\section{Installation Guidelines}
	
	Provide guidelines for the installation of the RS485 network:
	
	\begin{itemize}[label=--]
		\item Proper grounding techniques.
		\item Cable routing and management.
		\item Considerations for avoiding electromagnetic interference.
	\end{itemize}
	
	\section{Testing and Validation}
	
	Outline the procedures for testing and validating the RS485 network:
	
	\begin{itemize}[label=--]
		\item Functional testing of each device.
		\item Communication reliability testing.
		\item Troubleshooting steps.
	\end{itemize}
	
	\section{Maintenance and Support}
	
	Describe the maintenance procedures and support mechanisms for the RS485 network:
	
	\begin{itemize}[label=--]
		\item Regular checks and inspections.
		\item Firmware updates for devices.
		\item Contact information for technical support.
	\end{itemize}
	
	\section{Conclusion}
	
	Summarize the key points of the RS485 wired network specification and emphasize any critical considerations for successful implementation.
	
\end{document}
